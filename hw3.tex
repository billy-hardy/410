\documentclass[10pt]{article}


\usepackage{graphicx}
\usepackage{graphicx}
\usepackage{latexsym}
\PassOptionsToPackage{boxed,section}{algorithm}
\usepackage{algorithm}
\usepackage{algorithmic}
\usepackage{enumerate}
\usepackage{times}
\usepackage{url}
\usepackage{hyperref}
\hypersetup{colorlinks=true}
\usepackage{color}

\voffset=-.5in
\setlength{\textheight}{8.75in}
% \setlength{\columnsep}{2.0pc}
\setlength{\textwidth}{6.8in}
%\setlength{\footheight}{0.0in}
\setlength{\topmargin}{0.25in}
\setlength{\topmargin}{0.25in}
\setlength{\headheight}{0.0in}
\setlength{\headsep}{0.0in}
\setlength{\oddsidemargin}{-.19in}
\setlength{\parindent}{1pc}


%\input{defs.tex}
\title{Advanced Analysis of Algorithms - Homework III}

\author{
 Billy Hardy \\ LCSEE, \\
 West Virginia University, \\ Morgantown, WV \\
\mbox\{whardy2@mix.wvu.edu\}}

\begin{document}
\date{\today}
\maketitle
\bibliographystyle{alpha}

\section{Problems}
\begin{enumerate}

\item%Problem 1
\begin{verbatim}
int countInv(int A[], int start, int end) {
  if(start == end || start + 1 == end) {
    return 0;
  }
  int a, b;
  a = countInv(A, start, (start+end)/2);
  b = countInv(A, start+(start+end)/2, end);
  int i = start, j;
  int count = 0;
  while(i < (start+end)/2) {
    j = start+(start+end)/2;
    while(j < end) {
      if(A[i] > A[j]) {
        count++;
      }
      j++;
    }
    i++;
  }
  return a+b+count;
}
\end{verbatim}

\item%Problem 2

\begin{enumerate}[a)]

\item%a
The optimal substructure property does not hold when there are negative weight cycles.

\item%b
\begin{verbatim}
int floyd(int n, const float W[][], float D[][]) {
  int i, j, k;
  D = W; //do a deep copy, not shallow
  for(k=0; k<n; k++) {
    for(i=0; i<n; i++) {
      for(j=0; j<n; j++) {
        D[i][j] = min(D[i][j], D[i][k]+D[k][j]);
      }
    }
  }
  for(i=0; i<n; i++) {
    if(D[i][i] < 0) {
      return FALSE; //if path from i to i is negative, there has to be a negative cycle
    }
  }
  return TRUE;
}
\end{verbatim}

\end{enumerate}

\item%Problem 3

\begin{enumerate}[a)]

\item%a
\[ $X =$ \left( \begin{array}{cc}
$9$ & $3$ \\
$2$ & $-1$ \end{array} \right)\] 
\[ $Y =$ \left( \begin{array}{cc}
$1$ & $2$ \\
$2$ & $-1$ \end{array} \right)\]

For Strassen's matrix multiplication algorithm, we have to calculate matrices $m_1$ through $m_7$. 
\[\begin{array}{l}
$m_1 = (a_{11} + a_{22})(b_{11} + b_{22})$ \\
$m_2 = (a_{21} + a_{22})b_{11}$ \\
$m_3 = a_{11}(b_{12} - b_{22})$ \\
$m_4 = a_{22}(b_{21} - b_{11}$ \\
$m_5 = (a_{11} + a_{12})b_{22}$ \\
$m_6 = (a_{21} - a_{11})(b_{11} + b_{12})$ \\
$m_7 = (a_{12} - a{22})(b_{21} + b_{22})$ \end{array}\]

$X \times Y = C$ where \[ C = \left( \begin{array}{cc}
$m_1 + m_4 - m_5 + m_7$ & $m_3 + m_5$ \\
$m_2 + m_4$ & $m_1 + m_3 - m_2 + m_6$ \end{array} \right)\]

So, in this case, \[\begin{array}{l}
$m_1 = (9 + -1)(1 + -1) = 0$ \\
$m_2 = (2 + -1)(1) = 1$ \\
$m_3 = 9(2 - -1) = 27$ \\
$m_4 = -1(2 - 1) = -1$ \\
$m_5 = (9 + 3)(-1) = -12$ \\
$m_6 = (2 - 9)(1 + 2) = -21$ \\
$m_7 = (3 - -1)(2 + -1) = 4$ \end{array}\] and \[ $C =$ \left( \begin{array}{cc}
$0 + -1 - (-12) + 4$ & $27 + (-12)$ \\
$1 + -1$ & $0 + 27 - 1 + (-21)$ \end{array} \right) $=$
\left( \begin{array}{cc}
$15$ & $15$ \\
$0$ & $5$ \end{array} \right)\]

\item%b


\end{enumerate}

\item%Problem 4


\item%Problem 5

\begin{enumerate}[a)]

\item%a
For each binary tree of $n$ nodes, there is exactly one ordering of the distinct keys which yields a binary search tree. So this problem reduces to show that the number of binary trees of $n$ nodes is $\frac{1}{n+1}\binom{2n}{n}$
Let $T_n$ be the sequence which gives the number of binary trees of $n$ nodes. $T_n = \sum_{k=0}^{n-1} T_k \times T_{n-k-1}$ and $T_0 = 1$. This is because once we choose the root node, we either put $0$ nodes in the left subtree and $n-1$ in the right, $1$ in the left and $n-2$ in the right \ldots. So $T_n = \{1, 1, 2, 5, 14, 42, \ldots \} \forall n \greatereq 0$. Let $T(z) = \sum_{N=0}^{\infty} T_N z^N$. So $T(z) = T_0 + T_1z + T_2z^2 + T_3z^3 + \ldots$. Squaring both sides we obtain $\[T(z)\]^2 = (T_0T_0) + (T_0T_1 + T_1T_0)z + (T_0T_2 + T_1T_1 +T_2T_0)z^2 + \ldots$ and by the definition of $T_n$, we get $\[T(z)\]^2 = T_1 + T_2z +T_3z^2 + \ldots$. Next by algebra: $z\[T(z)\]^2 = T_1z +T_2z^2 + \ldots = -T_0 + T_0 + T_1z +T_2z^2 + \ldots = -T_0 + T(z)$. So $z\[T(z)\]^2 - T(z) + 1 = 0$ and by the quadratic formula, $T(z) = \frac{1 \pm \sqrt{1-4z}}{2z}$. Since $T(z) = T_0 = 1$, we can simplify this to $T(z) = \frac{1 - \sqrt{1-4z}}{2z}$ which goes to $1$ as $z \rightarrow 0$. Now expand $(1-4z)^{\frac{1}{2}} = 1 - {\frac{1}{2}}4z + \frac{(\frac{1}{2})(-\frac{1}{2})}{2!}(4z)^2 - \frac{(\frac{1}{2})(-\frac{1}{2})(-\frac{3}{2})}{3!}(4z)^3 + \frac{(\frac{1}{2})(-\frac{1}{2})(-\frac{3}{2})(-\frac{5}{2})}{4!}(4z)^4 - \ldots = 1 - 2z - \frac{1}{2!}4z^2 - \frac{3 \times 1}{3!}8z^3 - \frac{5 \times 3 \times 1}{4!}16z^4 - \ldots$. Plugging this into $T(z)$, we get $T(z) = 1 + \frac{1}{2!}2z + \frac{3 \times 1}{3!}4z^2 + \frac{5 \times 3 \times 1}{4!}8z^3 + \ldots = 1 + \frac{1}{2}(\frac{2!}{1!1!})z + \frac{1}{3}(\frac{4!}{2!2!})z^2 + \frac{1}{4}(\frac{6!}{3!3!}z^3 + \ldots = \sum_{k=0}{\infty}\frac{1}{k+1}\binom{2k}{k}z^k$. Therefore, $T_n = \frac{1}{n+1}\binom{2n}{n}$ as desired. 

\item%b


\end{enumerate}

\end{enumerate}

\end{document}
